\documentclass[8pt,notitlepage,letter]{report}

\usepackage{amsfonts}
\usepackage{amssymb}
\usepackage{amsthm}
\usepackage{amsmath,amscd}
\usepackage[all]{xy}
\usepackage{moreverb}
\usepackage{fancyhdr}
\usepackage{graphicx}


    \newtheorem{problem}{Problem}
    \newtheorem{theorem}{Theorem}[section]
    \newtheorem{lemma}[theorem]{Lemma}
    \newtheorem{proposition}[theorem]{Proposition}
    \newtheorem{corollary}[theorem]{Corollary}

    \newenvironment{solution}[1][Solution]{\begin{trivlist}
    \item[\hskip \labelsep {\bfseries #1}]}{\end{trivlist}}
    \newenvironment{definition}[1][Definition]{\begin{trivlist}
    \item[\hskip \labelsep {\bfseries #1}]}{\end{trivlist}}
    \newenvironment{example}[1][Example]{\begin{trivlist}
    \item[\hskip \labelsep {\bfseries #1}]}{\end{trivlist}}
    \newenvironment{remark}[1][Remark]{\begin{trivlist}
    \item[\hskip \labelsep {\bfseries #1}]}{\end{trivlist}}

  %  \newcommand{\qed}{\nobreak \ifvmode \relax \else
  %        \ifdim\lastskip<1.5em \hskip-\lastskip
  %        \hskip1.5em plus0em minus0.5em \fi \nobreak
  %        \vrule height0.75em width0.5em depth0.25em\fi}

    \renewcommand{\qedsymbol}{\textsquare}

\addtolength{\textwidth}{1in}
\addtolength{\hoffset}{-0.5in}
\addtolength{\textheight}{2in}
\addtolength{\voffset}{-0.3in}

\relpenalty=9999
\binoppenalty=9999

\newcommand{\PP}{\mathbb{P}}

\pagestyle{fancy}
%\fancyhead[CO,CE]{S.Chaichenets}
\fancyfoot[CO,CE]{S.Chaichenets}
\fancyfoot[RO, LE] {\thepage}


\begin{document}

\title{STAT 580 Stochastic Processes: HW 3}
\author{ S.\ Chaichenets }
\maketitle


%%%%%%%%%%%%%%%%%%%%%%%%%%%%%%%%%%%%%%%%%%%%

\begin{problem}
\small
\begin{equation}
\frac{1}{2^{2n}} {2n \choose n} = \frac{1}{2n} \prod_{j=1}^{n-1} \left(1+\frac{1}{2j}\right)
\qquad \forall n\geq 1
\end{equation}
\normalsize
\end{problem}

\begin{solution}
Induction step:

\begin{equation}
\begin{split}
\frac{1}{2^{2(n+1)}} {2(n+1) \choose n+1} 
	& := \frac{1}{2^{2n}}\frac{1}{2^2} \frac{(2n+2)!}{(n+1)!\,(n+1)!} 
	 = \left[ \frac{1}{2^{2n}}\frac{(2n)!}{n!n!} \right]
		\frac{(2n+1)(2n+2)}{(n+1)(n+1)}\frac{1}{2\times2} 			\\
	& = \left[ \frac{1}{2n}\prod_{j=1}^{n-1} \left(1+\frac{1}{2j}\right) \right]
		\frac{1}{2(n+1)}\frac{2n+1}{1} 
	 = \left[ \frac{1}{2(n+1)}\prod_{j=1}^{n-1} \left(1+\frac{1}{2j}\right) \right]
		\frac{2n+1}{2n}								\\
	& = \left[ \frac{1}{2(n+1)}\prod_{j=1}^{n-1} \left(1+\frac{1}{2j}\right) \right]
		\left( 1 + \frac{1}{2n} \right)
	 = \frac{1}{2(n+1)}\prod_{j=1}^{(n+1)-1} \left(1+\frac{1}{2j}\right)
\end{split}
\end{equation}
Induction base: interpreting an `empty' product as 1, we have, for $n=1$
$$
\frac{1}{2^2}{2 \choose 1} = \frac{1}{2\times2} \times 2 = \frac{1}{2}
$$
Now, for any $N\in \mathbb{N}$, we have
\begin{equation}
\begin{split}
\sum_{n=0}^{N} p_{2n}(0,0) &= \sum_n \frac{1}{2^n}\frac{1}{2^n}{2n \choose n}		\\
			&= \sum_n {\frac{1}{2n} \prod_{j=1}^{n-1} \left(1+\frac{1}{2j}\right)}	\\
			&\geq	\sum_{n}\frac{1}{2n}
\end{split}
\end{equation}
Since the harmonic series $\sum_n \frac{1}{2n}$ diverges, 
the sum $\sum_{n} p_{2n}(0,0)$ also diverges.

This means that the state $0$ in the symmetric one-dimensional random walk is transient,
and, since the only communicating class is $\mathbb{Z}$ itself\footnote{
	e.g. by translational symmetry
}, {\it any} state $n\in \mathbb{Z}$ is transient.
\qed
\end{solution}

\begin{problem}
For symmetric three-dimensional random walk, calculate the probabilities 
$p_{2n}(0,0)$, for $n\in\{1,2,3\}$.
\end{problem}
\begin{solution}{${\mathbf n=1}$}:\\
	There is a total of $6^2$ 2-step random walks in 3D, all equiprobable.
	We need to calculate how many of these end up in $0$.
	Note that the number of `+' and `-' steps must be balanced 
	for every dimension separately:
	$$ 2 = 0 + 0 + 2 = 0 + 2 + 0 = 2 + 0 + 0 $$
	In each of these, the order of + and - is immaterial: $+_x\,-_x = -_x\,+_x$.
	Thus
  	$$
		p_2(0,0) = \frac{2 \times 3}{6^2}  = \frac{1}{6}
	$$
\qed
\end{solution}

\begin{solution}${\mathbf n=2}$: \\
	We have $$ 4 = 0 + 0 + 4 = 0+4+0 = 4+0+0 = 0+2+2 = 2+0+2 = 2+2+0 $$
	The `one-dimensional' walks 4 steps in length 
	can happen in ${4\choose2}$ ways each. Therefore,
	$ {4\choose2} \times 3 + {2\choose1}\times{2\choose1}\times3 =18+12=30$ 
	out of $6^4$ walks return to the origin in the end:
	$$ p_4(0,0) = \frac{5\times6}{6^4} = \frac{5}{6^3} = \frac{5}{216} \approx 0.0231 $$
\qed
\end{solution}

\begin{solution}${\mathbf n=3}$: \\
	Again, 
	\scriptsize
	$$ 6 	= 0+0+6 
		= 0+6+0 
		= 6+0+0 
		= 0+4+2 
		= 4+0+2 
		= 4+2+0 
		= 0+2+4
		= 2+0+4
		= 2+4+0
		= 2+2+2
	$$
	\normalsize
	And the number of 6-segment paths returning to $0$ at the end is
	$$
		{6\choose3} \times 3 
		+ {4\choose2}\times{2\choose1}\times6 
		+ \left({2\choose1}\right)^3\times1
		= 60 + 72 + 8 = 140
	$$
	out of $6^6$ total.
	$$ p_4(0,0) = \frac{140}{6^6} = \frac{35}{11664} \approx 0.00300 $$
\qed
\end{solution}

\begin{problem}
Random walk on $S=\{0,1,\ldots,10\}$, starting at 5. $q_x=1/x$, $p_x = 1-q_x$ 
for $1\leq x \leq 9$. Find probaility of hitting 0 before hitting 10.
\end{problem}

\begin{solution}
As in the notes, we have
$$
a(y) - a(y-1) = \frac{p_y}{q_y}[a(y+1)-a(y)]
$$
$$
a(y) - a(y-1) = s_{z}\cdots s_{y}[a(z)-a(z-1)]
$$
Where $s_\omega := p_\omega/q_\omega$. Now, for $x\leq \omega \leq z$,
$$
	a(z)-a(\omega) = \sum_{\omega+1}^z a(y)-a(y-1)
		= \sum_{y=\omega+1}^{z} s_{y}\ldots s_z [a(z)-a(z-1)]
$$
Using the boundary condition $1 = a(z) - a(x) 
	= \sum_{y=x+1}^z s_y\cdots s_z [a(z)-a(z-1)]$,
we obtain

\begin{equation}
	a(\omega) = \frac{\sum_{y=\omega+1}^{z}s_y\ldots s_z}
			{\sum_{y=x+1}^{z}s_y\ldots s_z}
\end{equation}

Substituting the numbers in this particular example, we have:

\begin{tabular}{cccc}
x & $p_x$ & $q_x$ & $s_x$ \cr
1 & 0 & 1 & 0		\cr
2 & 1/2 & 1/2 & 1	\cr
3 & 2/3 & 1/3 & 2	\cr
4 & 3/4 & 1/4 & 3	\cr
5 & 4/5 & 1/5 & 4	\cr
6 & 5/6 & 1/6 & 5	\cr
7 & 6/7 & 1/7 & 6	\cr
8 & 7/8 & 1/8 & 7	\cr
9 & 8/9 & 1/9 & 8	\cr
\end{tabular}

hence $\sum_{y=6}^9 s_y \cdots s_9 = 5+6+7+8 + 6+7+8 + 7+8 + 8$, and
$\sum_{y=2}^9 s_y \cdots s_9  = 1+2+\cdots+8 + 2+3+\cdots+8 + \cdots + 8 $, and
we arrive at the answer $a(5) = \frac{35}{102}$.
\qed
\end{solution}

\begin{problem}
Branching process $p_0=\frac{1}{4}, p_1=\frac{1}{4}, p_2=\frac{1}{2}$. 
Find probability of extinction\footnote{
	I assume, starting from initial population of 1} by time 10.
\end{problem}
\begin{solution}

% Let us first \emph{estimate} the extinction probability by time 10:

The reproduction parameter $\mu=\frac{5}{4}$, and 
the total extinction probability $a$ is given by the smallest root of
\begin{equation}
\phi(a) = \sum_k a^k p_k = a
\end{equation}

In our case, this becomes
\begin{equation}
p_0 + (p_1 - 1) a + p_2 a^2 
	= \frac{1}{4} - \frac{3}{4} a + \frac{1}{2} a^2
	= \frac{1}{4}(a-1)(2a-1)
	= 0
\end{equation}
So that the overall extinction probability $a=\frac{1}{2}$. 
Let us now calculate\footnote{
	This week, I'm in love with {\rm SAGE}, http://www.sagemath.org} 
the first several iterates of $\phi^n(0)$:

\begin{tabular}{ccc}
n & $\phi^n(0)$ & $\frac{1}{2} -\phi^n(0)$ \\
\hline
1 & 0.25000 & 0.25000 			\\
2 & 0.34375 & 0.15625			\\
3 & 0.39501 &0.10498		\\
4 & 0.42677 & 0.07322		\\
5 & 0.44776 & 0.05223		\\
6 & 0.46218 & 0.03781		\\
7 & 0.47235 & 0.02764		\\
8 & 0.47964 & 0.02035		\\
9 & 0.48494 & 0.01506		\\
{\bf 10}& {\bf 0.48882}&  {\bf 0.011180}	\\
11 & 0.49167 & 0.008323	\\
12 & 0.49379 & 0.006206	\\
13 & 0.49536 & 0.004635	\\
\end{tabular}

We see that extinction probability within first 10 generations is 
$\approx 0.48882$.\marginpar{{\bf 4(a)}}

I could not think of a pretty picture to produce with simulations, 
but the code listed in Appendix produces probability of extinction 
within 0.1 of the `theoretical' value, for a run of 100 populations.


%Now, after 10 steps the expected population size is 
%$\left(\frac{5}{4}\right)^{10} \approx 9.3 > 9$, and 

%We have 
%$$
%a_10 := \PP(X_{n+1}=0 | X_0=1) = \phi(a_9) := \sum_{k=0}^{\infty} \PP(X_9=0 | X_0=1)^k p_k
%$$

\qed
\end{solution}
\newpage
\appendix
\section*{Code Listing: Branching Processes}

\verbatimtabinput[4]{hw3q4.py}

\end{document}
