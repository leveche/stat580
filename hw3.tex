\documentclass[8pt,notitlepage]{report}

\usepackage{amsfonts}
\usepackage{amssymb}
\usepackage{amsthm}
\usepackage{amsmath,amscd}
\usepackage[all]{xy}
\usepackage{moreverb}
\usepackage{fancyhdr}
\usepackage{graphicx}


    \newtheorem{problem}{Poblem}
    \newtheorem{theorem}{Theorem}[section]
    \newtheorem{lemma}[theorem]{Lemma}
    \newtheorem{proposition}[theorem]{Proposition}
    \newtheorem{corollary}[theorem]{Corollary}

    \newenvironment{solution}[1][Solution]{\begin{trivlist}
    \item[\hskip \labelsep {\bfseries #1}]}{\end{trivlist}}
    \newenvironment{definition}[1][Definition]{\begin{trivlist}
    \item[\hskip \labelsep {\bfseries #1}]}{\end{trivlist}}
    \newenvironment{example}[1][Example]{\begin{trivlist}
    \item[\hskip \labelsep {\bfseries #1}]}{\end{trivlist}}
    \newenvironment{remark}[1][Remark]{\begin{trivlist}
    \item[\hskip \labelsep {\bfseries #1}]}{\end{trivlist}}

  %  \newcommand{\qed}{\nobreak \ifvmode \relax \else
  %        \ifdim\lastskip<1.5em \hskip-\lastskip
  %        \hskip1.5em plus0em minus0.5em \fi \nobreak
  %        \vrule height0.75em width0.5em depth0.25em\fi}

    \renewcommand{\qedsymbol}{\textsquare}

\addtolength{\textwidth}{1in}
\addtolength{\hoffset}{-0.5in}
\addtolength{\textheight}{2in}
\addtolength{\voffset}{-0.3in}

\relpenalty=9999
\binoppenalty=9999

\newcommand{\PP}{\mathbb{P}}

\pagestyle{fancy}
%\fancyhead[CO,CE]{S.Chaichenets}
\fancyfoot[CO,CE]{S.Chaichenets}
\fancyfoot[RO, LE] {\thepage}


\begin{document}

\title{STAT 580 Stochastic Processes: HW 3}
\author{ S.\ Chaichenets }
\maketitle


%%%%%%%%%%%%%%%%%%%%%%%%%%%%%%%%%%%%%%%%%%%%

\begin{problem}
\small
\begin{equation}
\frac{1}{2^{2n}} {2n \choose n} = \frac{1}{2n} \prod_{j=1}^{n-1} \left(1+\frac{1}{2j}\right)
\qquad \forall n\geq 1
\end{equation}
\normalsize
\end{problem}

\begin{solution}
Induction step:

\begin{equation}
\begin{split}
\frac{1}{2^{2(n+1)}} {2(n+1) \choose n+1} 
	& := \frac{1}{2^{2n}}\frac{1}{2^2} \frac{(2n+2)!}{(n+1)!\,(n+1)!} 
	 = \left[ \frac{1}{2^{2n}}\frac{(2n)!}{n!n!} \right]
		\frac{(2n+1)(2n+2)}{(n+1)(n+1)}\frac{1}{2\times2} 			\\
	& = \left[ \frac{1}{2n}\prod_{j=1}^{n-1} \left(1+\frac{1}{2j}\right) \right]
		\frac{1}{2(n+1)}\frac{2n+1}{1} 
	 = \left[ \frac{1}{2(n+1)}\prod_{j=1}^{n-1} \left(1+\frac{1}{2j}\right) \right]
		\frac{2n+1}{2n}								\\
	& = \left[ \frac{1}{2(n+1)}\prod_{j=1}^{n-1} \left(1+\frac{1}{2j}\right) \right]
		\left( 1 + \frac{1}{2n} \right)
	 = \frac{1}{2(n+1)}\prod_{j=1}^{(n+1)-1} \left(1+\frac{1}{2j}\right)
\end{split}
\end{equation}
Induction base: interpreting an `empty' product as 1, we have, for $n=1$
$$
\frac{1}{2^2}{2 \choose 1} = \frac{1}{2\times2} \times 2 = \frac{1}{2}
$$
Now, for any $N\in \mathbb{N}$, we have
\begin{equation}
\begin{split}
\sum_{n=0}^{N} p_{2n}(0,0) &= \sum_n \frac{1}{2^n}\frac{1}{2^n}{2n \choose n}		\\
			&= \frac{1}{2n} \prod_{j=1}^{n-1} \left(1+\frac{1}{2j}\right)	\\
			&\geq	\sum_{n}\frac{1}{2n}
\end{split}
\end{equation}
Since the harmonic series $\sum_n \frac{1}{2n}$ diverges, 
the sum $\sum_{n} p_{2n}(0,0)$ also diverges.

This means that the state $0$ in the symmetric one-dimensional random walk is transient,
and, since the only communicating class is $\mathbb{Z}$ itself\footnote{
	or by translational symmetry
}, {\it any} state $n\in \mathbb{Z}$ is transient.
\qed
\end{solution}

\begin{problem}
For symmetric three-dimensional random walk, calculate the probabilities 
$p_{2n}(0,0)$, for $n\in\{1,2,3\}$.
\end{problem}
\begin{solution}{${\mathbf n=1}$}:\\
	There is a total of $6^2$ 2-step random walks in 3D, all equiprobable.
	We need to calculate how many of these end up in $0$.
	Note that the number of `+' and `-' steps must be balanced 
	for every dimension separately:
	$$ 2 = 0 + 0 + 2 = 0 + 2 + 0 = 2 + 0 + 0 $$
	In each of these, the order of + and - is immaterial: $+_x\,-_x = -_x\,+_x$.
	Thus
  	$$
		p_2(0,0) = \frac{2 \times 3}{6^2}  = \frac{1}{6}
	$$
\qed
\end{solution}

\begin{solution}${\mathbf n=2}$: \\
	We have $$ 4 = 0 + 0 + 4 = 0+4+0 = 4+0+0 = 0+2+2 = 2+0+2 = 2+2+0 $$
	The `one-dimensional' walks 4 steps in length 
	can happen in ${4\choose2}$ ways each. Therefore,
	$ {4\choose2} \times 3 + {2\choose1}\times{2\choose1}\times3 =18+12=30$ 
	out of $6^4$ walks return to the origin in the end:
	$$ p_4(0,0) = \frac{5\times6}{6^4} = \frac{5}{6^3} = \frac{5}{216} \approx 0.0231 $$
\qed
\end{solution}

\begin{solution}${\mathbf n=3}$: \\
	Again, 
	\scriptsize
	$$ 6 	= 0+0+6 
		= 0+6+0 
		= 6+0+0 
		= 0+4+2 
		= 4+0+2 
		= 4+2+0 
		= 0+2+4
		= 2+0+4
		= 2+4+0
		= 2+2+2
	$$
	\normalsize
	And the number of 6-segment paths returning to $0$ at the end is
	$$
		{6\choose3} \times 3 
		+ {4\choose2}\times{2\choose1}\times6 
		+ \left({2\choose1}\right)^3\times1
		= 60 + 72 + 8 = 140
	$$
	out of $6^6$ total.
	$$ p_4(0,0) = \frac{140}{6^6} = \frac{35}{11664} \approx 0.00300 $$
\qed
\end{solution}

\newpage
\appendix
\section*{Code Listing: Random Walk on a Graph}

% \verbatimtabinput[4]{hw1q3.py}

\end{document}
