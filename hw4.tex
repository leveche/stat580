\documentclass[aps,prl,twocolumn,floatfix,letterpaper]{revtex4}

\usepackage{amsfonts}
\usepackage{amssymb}
\usepackage{amsthm}
\usepackage{amsmath,amscd}
\usepackage[all]{xy}
\usepackage{moreverb}
\usepackage{fancyhdr}
\usepackage{graphicx}


    \newtheorem{problem}{Problem}
    \newtheorem{theorem}{Theorem}[section]
    \newtheorem{lemma}[theorem]{Lemma}
    \newtheorem{proposition}[theorem]{Proposition}
    \newtheorem{corollary}[theorem]{Corollary}

    \newenvironment{solution}[1][Solution]{\begin{trivlist}
    \item[\hskip \labelsep {\bfseries #1}]}{\end{trivlist}}
    \newenvironment{definition}[1][Definition]{\begin{trivlist}
    \item[\hskip \labelsep {\bfseries #1}]}{\end{trivlist}}
    \newenvironment{example}[1][Example]{\begin{trivlist}
    \item[\hskip \labelsep {\bfseries #1}]}{\end{trivlist}}
    \newenvironment{remark}[1][Remark]{\begin{trivlist}
    \item[\hskip \labelsep {\bfseries #1}]}{\end{trivlist}}

  %  \newcommand{\qed}{\nobreak \ifvmode \relax \else
  %        \ifdim\lastskip<1.5em \hskip-\lastskip
  %        \hskip1.5em plus0em minus0.5em \fi \nobreak
  %        \vrule height0.75em width0.5em depth0.25em\fi}

    \renewcommand{\qedsymbol}{\textsquare}

\addtolength{\textwidth}{0.5in}
\addtolength{\hoffset}{-0.3in}
\addtolength{\textheight}{0.5in}
\addtolength{\voffset}{-0.3in}

\relpenalty=9999
\binoppenalty=9999

\newcommand{\PP}{\mathbb{P}}
\newcommand{\EE}{\mathbb{E}}

\pagestyle{fancy}
%\fancyhead[CO,CE]{S.Chaichenets}
\fancyfoot[CO,CE]{S.Chaichenets}
\fancyfoot[RO, LE] {\thepage}


\begin{document}

\title{STAT 580 Stochastic Processes \\
	Assignment 4}
\author{ S.\ Chaichenets }
\date{October 22, 2009}
\maketitle


%%%%%%%%%%%%%%%%%%%%%%%%%%%%%%%%%%%%%%%%%%%%

\begin{problem}
Symmetric random walk on $S=\{0,1,2,\ldots,10\}$ with absorbing boundaries.
Find the optimal stopping rule and the value function, for the payoff function
$$
	f(x) = (2,8,0,10,0,1,4,2,9,2,6)^T
$$
\end{problem}

\begin{solution}
For a symmetric random walk, a harmonic function is of the
form $h(x) = P\,h(x) = \alpha x$ %\marginpar{proof!}

We thus start with $$\varepsilon \supseteq \{ {\rm absorbing\ states} \} \cup \{ {\rm global\ maximum}\} = \{0,3,10\}$$
and continue by dissecting the set into intervals left and right of the maximum: \\*
${1} \in \varepsilon $ because $ 8 > 2 + \frac{10-2}{3-0}\times1 < 5$.\\*
${8} \in \varepsilon $ because $ 9 > 6 + \frac{10-6}{10-3}\times1 < 7$. \\*
Continuing the dissection until the entire set is exhausted, we obtain
$$\varepsilon=\{0,1,3,8,10\}$$, and the value function
$$
v = (0,8,9,10,9\frac{4}{5},9\frac{3}{5},9\frac{2}{5},9\frac{1}{5},9,7\frac{1}{2},6)^T
$$

The listing in the Appendix gives the code and a {\rm SAGE} interactive session used to obtain 
iterative approximations for $v$. After 20 iterations, we obtained
\tiny
\begin{equation}
\begin{split}
u_{20} &= 
(2, 8, 9, 10, \frac{2630897439}{268435456}, \frac{1288720159}{134217728},
\frac{1261832693}{134217728}, \frac{1234945227}{134217728}, 9, 15/2, 6) \\
&\approx 
(2, 8, 9, 10, 9.8, 9.6, 9.4, 9.2, 9, 7.5, 6)
\end{split}
\end{equation}
\normalsize
\qed
\end{solution}

\begin{problem}
Find optimal strategy and the value of the following dice game:
1) Roll 3 dice, and either accept the lowest score as payoff, or
2) roll 2 dice, either accept the lowest score or
3) roll 1 die, the score determining the payoff.
\end{problem}

\begin{solution}
Let us model this game as a Markov chain on the state space
$S=\{*\} \cup \{(n,D) | n=0,1,2 {\rm \ and\ } D \in \{1,\ldots,6 \} \}$, 
where * denotes the `start' state (with zero payoff), $n$ is the
number of throws still available, and $D$ is the score on the dice.

the value function for $n=0$ coincides with the payoff:
$$
v(n=0,D) = D
$$

For $n=1$, we either accept the current score $D$, or roll one 
die to get the no-appeal payoff. The expectation of the latter is $3.5$, 
hence the value function\footnote{with the conventional ordering of states $(0,1,\ldots,6)$} 
in this case is 
$$
v(n=1) = (7/2,7/2,7/2,4,5,6)^T
$$

Let us now calculate the order statistic for two dice:
We have $\PP(D > n) = \frac{6-n}{6}; \ n=1,\ldots,6$, so that 
\scriptsize
\begin{equation}
\begin{split}
\PP(D_1 \wedge D_2 = n) &= \PP(D_1 = D_2 = n)
		+ {2\choose 1} \PP(D_1 = n) \times \PP(D_2 > n) 	\\
	&= \frac{1}{6^2} 
		+ \frac{1}{3}\times \frac{6-n}{6}			\\
	&= \frac{1}{36} (11,9,7,5,3,1)
\end{split}
\end{equation}
\normalsize

The expectation of $v(n=1)$ under this random variable is
\scriptsize
\begin{equation}
\begin{split}
\EE(v(n=1)) &= \frac{7}{2}\times\frac{11}{36}
	+ \frac{7}{2}\times\frac{9}{36}
	+ \frac{7}{2}\times\frac{7}{36}
	+ 4\times\frac{5}{36}
	+ 5\times\frac{3}{36}
	+ 6\times\frac{1}{36}		\\
	&= \frac{271}{72} \approx 3.76
\end{split}
\end{equation}
\normalsize

From which follows the value function for $n=2$:
$$
v(n=2) = (\frac{271}{72},\frac{271}{72},\frac{271}{72},4,5,6)^T
$$

Now, repeat the above step with order statistic for three dice:
\begin{widetext}
\begin{equation}
\begin{split}
\PP(D_1 \wedge D_2 \wedge D_3 = n) &= \PP(D_1=D_2=D_3=n) 	
				+ {3\choose2} \PP (D_1=D_2=n<D_3)
				+ {3\choose1} \PP (D_1=n<D_2,D_3)		\\
	&= \frac{1}{6^3} 
		+ 3 \times \frac{1}{6^2} \times \frac{6-n}{6}
		+ \frac{1}{2} \times \left(\frac{6-n}{6}\right)^2
	= \frac{1}{6^3} (91,61,37,19,7,1)	\\
\end{split}
\end{equation}
\end{widetext}

And finally, calculate the value of the entire game:
\begin{equation}
\begin{split}
\EE(*) &= \frac{1}{216}\left(91,61,37,19,7,1\right)
		\left(\frac{271}{72},\frac{271}{72},\frac{271}{72},4,5,6 \right)^T \\
	&= \frac{2209}{576} \approx 3.835
\end{split}
\end{equation}

We summarize the above results in the following table:

\begin{center}
\begin{tabular}{c| cccccc}
n \textbackslash\  d &  1	&  2	&  3	&  4  	&   5	&  6	\cr
\hline
* & \multicolumn{6}{c} {$\frac{2209}{576}$} \cr
2	& $\frac{271}{72}$ & $\frac{271}{72}$ & $\frac{271}{72}$ &	4 & 5 & 6	\cr
1	& $\frac{7}{2}$ &  $\frac{7}{2}$  &  $\frac{7}{2}$  & 4 & 5 &	6		\cr
0	& 1 & 2 & 3 & 4 & 5 & 6 
\end{tabular}
\end{center}

The optimal strategy is to keep playing, for as long as possible, until the minimum 
of a dice throw is 4 or greater.

\qed
\end{solution}

\pagebreak[3]

\begin{problem}
Random walk on a $3\times3$ board, starting in position $(11)$, with $(31)=:A$ 
the only absorbing state, and the payoff function
\scriptsize $f(xy) = \begin{cases} 75 & xy=(13) \\ 100 & xy=(33) \\ 0 & {\rm otherwise} \end{cases} $\normalsize. 
Find the value of the game and the optimal strategy.
\end{problem}

\begin{solution}
First of all, the `termination' set $\varepsilon \supseteq \{ A, 33 \} $, and clearly 
$ f^{-1}(0) \cap \varepsilon = \{A\} $. 
Thus we only have to decide whether $(13) \in \varepsilon$ or not.
To this end, calculate the probability of reaching $A$ before $(33)$,starting from $(31)$: 
%- which by symmetry is the same as reaching $(33)$, before $(31)$, starting from $(11)$:
the transition matrix for the board is
\small
\begin{equation}
 P = 
 \bordermatrix{
	& \mathbf{11} & \mathbf{12} & \mathbf{13} & \mathbf{21} & \mathbf{22} & \mathbf{23} & \mathbf{31} & \mathbf{32} & \mathbf{33} \cr
\mathbf{11} & 	0	& \frac{1}{2}	& 0 	& \frac{1}{2} & 0 & 0 & 0 & 0 & 0	\cr
\mathbf{12} & \frac{1}{3} & 0 & \frac{1}{3} & 0 & \frac{1}{3} & 0 & 0 & 0 & 0		\cr
\mathbf{13} & 0 & \frac{1}{2} & 0 & 0 & 0 & \frac{1}{2} & 0 & 0 & 0			\cr
\mathbf{21} & \frac{1}{3} & 0 & 0 & 0 & \frac{1}{3} & 0 & \frac{1}{3} & 0 & 0		\cr
\mathbf{22} & 0 & \frac{1}{4} & 0 & \frac{1}{4} & 0 & \frac{1}{4} & 0 & \frac{1}{4} & 0 \cr
\mathbf{23} & 0 & 0 & \frac{1}{3} & 0 & \frac{1}{3} & 0 & 0 & 0 & \frac{1}{3}		\cr
\mathbf{31} & 0 & 0 & \frac{1}{2} & 0 & 0 & 0 & 0 & \frac{1}{2} & 0			\cr
\mathbf{32} & 0 & 0 & 0 & 0 & \frac{1}{3} & 0 & \frac{1}{3} & 0 & \frac{1}{3}		\cr
\mathbf{33} & 0 & 0 & 0 & 0 & 0 & \frac{1}{2} & 0 & \frac{1}{2} & 0
 }
\end{equation}
\normalsize
Take $Q$ to be the submatrix of transitions inside the set $\{11,12,21,22,23,31,32\}$, so that
\scriptsize
\begin{equation}
 (I-Q)^{-1} = 
 \bordermatrix{
& \mathbf{11} & \mathbf{12} & \mathbf{21} & \mathbf{22} & \mathbf{23} & \mathbf{31} & \mathbf{32} \cr
\mathbf{11} & \frac{114}{65} & \frac{147}{130} & \frac{147}{130} & \frac{66}{65} & \frac{33}{130} & \frac{36}{65} & \frac{69}{130}\cr
\mathbf{12} & \frac{48}{65} & \frac{209}{130} & \frac{79}{130} & \frac{62}{65} & \frac{31}{130} & \frac{22}{65} & \frac{53}{130}\cr 
\mathbf{21} &\frac{10}{13} & \frac{17}{26} & \frac{43}{26} & \frac{14}{13} & \frac{7}{26} & \frac{10}{13} & \frac{17}{26}\cr 
\mathbf{22} & \frac{6}{13} & \frac{9}{13} & \frac{9}{13} & \frac{24}{13} & \frac{6}{13} & \frac{6}{13} & \frac{9}{13}\cr
\mathbf{23} & \frac{2}{13} & \frac{3}{13} & \frac{3}{13} & \frac{8}{13} & \frac{15}{13} & \frac{2}{13} & \frac{3}{13}\cr
\mathbf{31} &\frac{6}{65} & \frac{9}{65} & \frac{9}{65} & \frac{24}{65} & \frac{6}{65} & \frac{84}{65} & \frac{48}{65}\cr 
\mathbf{32} &\frac{12}{65} & \frac{18}{65} & \frac{18}{65} & \frac{48}{65} & \frac{12}{65} & \frac{38}{65} & \frac{96}{65}
}
\end{equation}
\normalsize
and
$$
S^T = \bordermatrix{
& \mathbf{11} & \mathbf{12} & \mathbf{21} & \mathbf{22} & \mathbf{23} & \mathbf{31} & \mathbf{32}\cr
\mathbf{13} & 0 & \frac{1}{3} & 0 & 0 & \frac{1}{3} & \frac{1}{2} & 0\cr 
\mathbf{33} & 0 & 0 & 0 & 0 & \frac{1}{3} & 0 & \frac{1}{3}
}
$$
Hence
\small
\begin{equation}
\left((I-Q)^{-1}S\right)^T = \bordermatrix{
& \mathbf{11} & \mathbf{12} & \mathbf{21} & \mathbf{22} & \mathbf{23} & \mathbf{31} & \mathbf{32}\cr
\mathbf{13} & \frac{48}{65} & \frac{51}{65} & \frac{9}{13} & \frac{8}{13} & \frac{7}{13} & \frac{47}{65} & \frac{29}{65}\cr 
\mathbf{33} & \frac{17}{65} & \frac{14}{65} & \frac{4}{13} & \frac{5}{13} & \frac{6}{13} & \frac{18}{65} & \frac{36}{65}
}
\end{equation}
\normalsize
From which we read off $\PP((31) \to (33) {\rm \ before\ } A) = \frac{47}{65} \approx 0.723 < \frac{75}{100}$.
\marginpar{this prob should be \frac{3}{5}}
Thus the expected payoff for continuing the game after reaching the state $(31)$ is smaller than the payoff 
for stopping and collecting $75$.

Thus \begin{equation} \varepsilon = \{A,(31),(33)\} \end{equation} and the optimal strategy 
is to keep playing until hitting either of these states.

Int the appendix, we repeat the calculation above to see the probability distribution for first hit of the set $\varepsilon$:
$$
  \left(\frac{3}{7},\frac{3}{7},\frac{1}{7}\right)
$$
Combining the results above, we obtain the value of the game, starting at position $(11)$, as 
the expected payoff:
$$ 
	\EE_{11} = \frac{3}{7}\times(75+0) + 100\times\frac{1}{7} = \frac{325}{7} \approx 46.43
$$
\qed
\end{solution}

\appendix
%\section*{Code Listing}
%
%scriptsize
%
%\normalsize
\end{document}

