\documentclass[8pt,notitlepage]{report}

\usepackage{amsfonts}
\usepackage{amssymb}
\usepackage{amsthm}
\usepackage{amsmath,amscd}
\usepackage[all]{xy}
\usepackage{moreverb}
\usepackage{fancyhdr}

    \newtheorem{problem}{Poblem}
    \newtheorem{theorem}{Theorem}[section]
    \newtheorem{lemma}[theorem]{Lemma}
    \newtheorem{proposition}[theorem]{Proposition}
    \newtheorem{corollary}[theorem]{Corollary}

    \newenvironment{solution}[1][Solution]{\begin{trivlist}
    \item[\hskip \labelsep {\bfseries #1}]}{\end{trivlist}}
    \newenvironment{definition}[1][Definition]{\begin{trivlist}
    \item[\hskip \labelsep {\bfseries #1}]}{\end{trivlist}}
    \newenvironment{example}[1][Example]{\begin{trivlist}
    \item[\hskip \labelsep {\bfseries #1}]}{\end{trivlist}}
    \newenvironment{remark}[1][Remark]{\begin{trivlist}
    \item[\hskip \labelsep {\bfseries #1}]}{\end{trivlist}}

  %  \newcommand{\qed}{\nobreak \ifvmode \relax \else
  %        \ifdim\lastskip<1.5em \hskip-\lastskip
  %        \hskip1.5em plus0em minus0.5em \fi \nobreak
  %        \vrule height0.75em width0.5em depth0.25em\fi}

    \renewcommand{\qedsymbol}{\textsquare}

\addtolength{\textwidth}{1in}
\addtolength{\hoffset}{-0.5in}
\addtolength{\textheight}{2in}
\addtolength{\voffset}{-0.3in}

\newcommand{\PP}{\mathbb{P}}

\pagestyle{fancy}
%\fancyhead[CO,CE]{S.Chaichenets}
\fancyfoot[CO,CE]{S.Chaichenets}
\fancyfoot[RO, LE] {\thepage}


\begin{document}

\title{STAT 580 Stochastic Processes: HW 1}
\author{ S.\ Chaichenets }
\maketitle


%%%%%%%%%%%%%%%%%%%%%%%%%%%%%%%%%%%%%%%%%%%%

\begin{problem}
	$(X_n)_0^\infty$ (time-homogeneous) Markov chain with transition matrix $P$  $\implies$
	$(X_{2n})$ is also a Markov chain, with transition matrix $P^2$.
\end{problem}

\begin{solution}
% Let $ Y := ({X_n}_j) $ be a sub-sequence of $X := {(X_n)}^{\infty}_{n=0}$

Observe that the ``law of total probability'' gives
\begin{equation}
\small
\PP(X_{n+2}=x_{n+2}|X_{n}=x_{n})
	=  
\sum_{x_{n+1} \in S} \PP(X_{n+2}=x_{n+2}|X_{n+1}=x_{n+1}) \ \PP(X_{n+1}=x_{n+1}|X_{n}=x_{n}) \ \PP(X_{n}=x_{n})
\end{equation}
\normalsize

Since $(X_n)$ is a time-homogenous Markov chain, we obtain, for a probability vector $\phi_n$:

\begin{equation}
% \begin{split}
\phi_{n+2} | \phi_{n} := 
\PP(X_{n+2}=x_{n+2}|X_{n}=x_{n})_{x_n,x_{n+2} \in S}
	= 
	\sum_{i,j \in S} \phi_n(i) p(ij) p(jk)
	= \phi_n P^2
% \end{split}
\end{equation}

Which is to say, $P^2$ is the transition matrix of $(X_{2n})$. In our construction, 
we have not used any assumption on any chain state $\phi_{m < n}$ prior to $n$,
which indicates that the Markovian property holds for the sub-chain as well.\footnote{This can be easily turned into a formal argument along the lines $(X_{2n})$ not Markovian $\implies$ $(X_n)$ not Markovian.}

\end{solution}

%%%%%%%%%%%%%%%%%%%%%%%%%%%%%%%%%%%%%%%%%%%%

\begin{problem}
  Let $(X_n)_0^\infty$ be a sequence of iid r.v.'s such that $$ \PP(X=0) = \PP(X=1) = \frac{1}{2} $$
  Define stopping time $ T := 1 + 1_{\{X_0 = 1\} } $, and a stopped sequence $ (Y_n) := X_{n \wedge T} $.

  {\bf Claim:} $(Y_n)$ does not have a Markov property.
\end{problem}

\begin{solution}

The value of $Y_0 = X_0$ fixes $T$ and thus the value of all $Y_{n \geq 2}$.
% whereas the value of $Y_{n-1}$ is irrelevant to the value of $Y_{n}$ for $n > 2$

\xymatrix{
	X_0 \ar[r] 	& X_1 \ar[r]^{X_0=0}_{T=1} \ar[dr]^{X_0=1}_{T=2}	& X_1 \ar[r] 	& \ldots \\
			& 			& X_2 \ar[r]	& X_2 \ar[r] & \ldots
}

Formally, we have
$$
 T~\ge~1~\implies~Y_0~=~X_0,\, Y_1 = X_1
$$
and 
\small
$$
 X_0 = 0 \iff Y_0 = 0 \iff T=1 \implies n \wedge T = 1 \  \forall n \geq 1 ;\   Y_{n\geq 1} = Y_1 = X_1
$$
\normalsize
similarly,
\small
$$
 X_0 = 1 \iff Y_0 = 2 \iff T=2 \implies n \wedge T = 2 \  \forall n \geq 2 ;\  Y_{n\geq 2} = Y_2 = X_2,\ Y_1 = X_1
$$
\normalsize
Thus
$$
\PP(Y_2=0\,|\,Y_1=Y_0=0) = \PP(Y_2=X_1=0\,|\,Y_1=X_1=Y_0=X_0=0) = 1
$$
whereas
\small
$$
\PP(Y_2=0\,|\,Y_1=0,Y_0=1) = \PP(Y_2=X_2=0\,|\,Y_1=X_1=0,\,X_0=1) = \PP(X_2=0\,|\,X_0=x_0,X_1=x_1) = \frac{1}{2}
$$
\normalsize

This is a direct contradiction to the Markov property, which demands, {\it inter caetera}, that
$$
 \PP(Y_2=0\, |\, Y_1=Y_0=0 ) = \PP(Y_2=0\,|\, Y_1=0,\,Y_0\neq0) = \PP(Y_2=0\,|\,Y_1=0)
$$

\qed
\end{solution}

%%%%%%%%%%%%%%%%%%%%%%%%%%%%%%%%%%%%%%%%%%%%

\begin{problem} Random walk on a graph.

\end{problem}

\begin{solution}

Let us order the state space as $(00,01,10,02,11,20),$ the transition matrix for the given graph is then
$$
P = 
\bordermatrix{
  & \mathbf{00} & \mathbf{01} & \mathbf{10} &\mathbf{02} &\mathbf{11} &\mathbf{20} \cr
\mathbf{00} & 0  & \frac{1}{2}  &  \frac{1}{2} & 0  & 0  & 0 \cr
\mathbf{01} &  \frac{1}{4}  & 0  &  \frac{1}{4} & \frac{1}{4}  & \frac{1}{4}  & 0 \cr
\mathbf{10} & \frac{1}{4}  & \frac{1}{4}  &  0 & 0  & \frac{1}{4}  & \frac{1}{4} \cr
\mathbf{02} & 0  & \frac{1}{2}  &  0 & 0  & \frac{1}{2}  & 0 \cr
\mathbf{11} & 0  & \frac{1}{4}  &  \frac{1}{4} & \frac{1}{4}  & 0  & \frac{1}{4} \cr
\mathbf{20} & 0  & 0  &  \frac{1}{2} & 0  & \frac{1}{2}  & 0 \cr
}
=
\frac{1}{4}
\left(
\begin{array}{cccccc}
 0  & 2  &  2 & 0  & 0  & 0 \\
 1  & 0  &  1 & 1  & 1  & 0 \\
 1  & 1  &  0 & 0  & 1  & 1 \\
 0  & 2  &  0 & 0  & 2  & 0 \\
 0  & 1  &  1 & 1  & 0  & 1 \\
 0  & 0  &  2 & 0  & 2  & 0 \\
\end{array}
\right)
$$

\marginpar{{\bf 3(c)(3)}}

Before proceeding to numerical experiments, let us determine an invariant probability vector 
$\mathbf{\pi} = (\pi_{00},\pi_{01},\pi_{10},\pi_{02},\pi_{11},\pi_{20})$
from symmetry considerations: (00\, 02\, 20) (10\, 11\, 01) is clearly one of the symmetries of our graph,
and it already forces $\pi_{00} = \pi_{02} = \pi_{20} =: a; \ \pi_{01} = \pi_{11} = \pi_{10} =: b$
Moreover, the `mid-point' nodes $01,\,11,\,10$ are `twice as connected' as the end-nodes, so we expect $ b = 2a $.
In conjunction with the probability normalization $ 3(a+b) = 1 $, we obtain $ a = \frac{1}{9}, b = \frac{2}{9} $,
thus $$\mathbf{\pi} = \frac{1}{9} ( 1,2,2,1,2,1 ) $$


Simple calculation shows that $P^2 > 0$, therefore the chain is ergodic and the above invariant probability 
vector is unique.

We now proceed to show, using numerical computation\footnote{I prefer a free general-purpose language {\rm Python}, 
which together with ScipPy and NumPy libraries gives any computer algebra package like Matlab or Maple 
a run for its money.},
that indeed $\pi = \pi\,P$:

\begin{verbatimtab}[4]
Python 2.5 (r25:51918, Sep 19 2006, 08:49:13) 
[GCC 4.0.1 (Apple Computer, Inc. build 5341)] on darwin
IDLE 1.2      
>>> from scipy import *
>>> P = array ( ( (0,2,2,0,0,0),\
		(1,0,1,1,1,0),\
		(1,1,0,0,1,1),\
		(0,2,0,0,2,0),\
		(0,1,1,1,0,1),\
		(0,0,2,0,2,0) ) )
>>> p = (1./9.,2./9.,2./9.,1./9.,2./9.,1./9.)
>>> M = diag ((1,1,1,1,1,1))
>>> for i in range (4):
		M = dot(M,P/4.)
		print M
		print M[0][0]
		o = dot (p,M)
\end{verbatimtab}

The above snipplet computes and prints $P^i$ for $i \in \{1..4\}$, together with 
$P^i_{00} (00\,00)$ and $\pi P^i$:

\small
\begin{verbatimtab}[4]
[[ 0.    0.5   0.5   0.    0.    0.  ]
 [ 0.25  0.    0.25  0.25  0.25  0.  ]
 [ 0.25  0.25  0.    0.    0.25  0.25]
 [ 0.    0.5   0.    0.    0.5   0.  ]
 [ 0.    0.25  0.25  0.25  0.    0.25]
 [ 0.    0.    0.5   0.    0.5   0.  ]]
0.0
[ 0.11111111  0.22222222  0.22222222  0.11111111  0.22222222  0.11111111] 

[[ 0.25    0.125   0.125   0.125   0.25    0.125 ]
 [ 0.0625  0.375   0.1875  0.0625  0.1875  0.125 ]
 [ 0.0625  0.1875  0.375   0.125   0.1875  0.0625]
 [ 0.125   0.125   0.25    0.25    0.125   0.125 ]
 [ 0.125   0.1875  0.1875  0.0625  0.375   0.0625]
 [ 0.125   0.25    0.125   0.125   0.125   0.25  ]]
0.25
[ 0.11111111  0.22222222  0.22222222  0.11111111  0.22222222  0.11111111]

[[ 0.0625    0.28125   0.28125   0.09375   0.1875    0.09375 ]
 [ 0.140625  0.15625   0.234375  0.140625  0.234375  0.09375 ]
 [ 0.140625  0.234375  0.15625   0.09375   0.234375  0.140625]
 [ 0.09375   0.28125   0.1875    0.0625    0.28125   0.09375 ]
 [ 0.09375   0.234375  0.234375  0.140625  0.15625   0.140625]
 [ 0.09375   0.1875    0.28125   0.09375   0.28125   0.0625  ]]
0.0625
[ 0.11111111  0.22222222  0.22222222  0.11111111  0.22222222  0.11111111] 

[[ 0.140625    0.1953125   0.1953125   0.1171875   0.234375    0.1171875 ]
 [ 0.09765625  0.2578125   0.21484375  0.09765625  0.21484375  0.1171875 ]
 [ 0.09765625  0.21484375  0.2578125   0.1171875   0.21484375  0.09765625]
 [ 0.1171875   0.1953125   0.234375    0.140625    0.1953125   0.1171875 ]
 [ 0.1171875   0.21484375  0.21484375  0.09765625  0.2578125   0.09765625]
 [ 0.1171875   0.234375    0.1953125   0.1171875   0.1953125   0.140625  ]]
0.140625
[ 0.11111111  0.22222222  0.22222222  0.11111111  0.22222222  0.11111111]
\end{verbatimtab}
\normalsize

We now calculate some higher powers of $P$, here I only give one example $P^{100}$:

\small
\begin{verbatimtab}[4]
>>> for i in range(4,100):
		M = dot(M,P/4.)
>>> print M, M.sum(axis=1)

[[ 0.11111111  0.22222222  0.22222222  0.11111111  0.22222222  0.11111111]
 [ 0.11111111  0.22222222  0.22222222  0.11111111  0.22222222  0.11111111]
 [ 0.11111111  0.22222222  0.22222222  0.11111111  0.22222222  0.11111111]
 [ 0.11111111  0.22222222  0.22222222  0.11111111  0.22222222  0.11111111]
 [ 0.11111111  0.22222222  0.22222222  0.11111111  0.22222222  0.11111111]
 [ 0.11111111  0.22222222  0.22222222  0.11111111  0.22222222  0.11111111]]
[ 1.  1.  1.  1.  1.  1.]
\end{verbatimtab}
\normalsize

These experiments convince us that $\pi = \frac{1}{9} ( 1,2,2,1,2,1 ) \approx 0.11111111 \times (1,2,2,1,2,1) $
is indeed the invariant probability vector. \marginpar{ {\bf 3(c)(1)} }

Finally, let us compute left egenvalues and associated eigenvectors of $P$:
\marginpar{ {\bf 3(c)(2)} }
\begin{verbatimtab}[4]
>>> scipy.linalg.eig(P/4.,left=True,right=False)
(array([-0.50+0.j,  0.25+0.j,  1.00+0.j, -0.50+0.j,  0.25+0.j, -0.50+0.j]),
array([[  4.69160878e-01,  -5.46383944e-01,   2.58198890e-01,
          0.00000000e+00,  -0.00000000e+00,   0.00000000e+00],
       [ -2.87974695e-01,  -4.34742087e-01,   5.16397779e-01,
          1.65937722e-01,   5.00000000e-01,  -5.34522484e-01],
       [ -6.50347060e-01,  -1.11641857e-01,   5.16397779e-01,
         -1.65937722e-01,  -5.00000000e-01,   5.34522484e-01],
       [  2.41018718e-01,   1.11641857e-01,   2.58198890e-01,
          3.10977345e-01,   5.00000000e-01,   4.85479132e-01],
       [ -1.94062740e-01,   5.46383944e-01,   5.16397779e-01,
         -7.87892413e-01,   1.20356739e-16,  -4.36435780e-01],
       [  4.22204900e-01,   4.34742087e-01,   2.58198890e-01,
          4.76915067e-01,  -5.00000000e-01,  -4.90433517e-02]]))
\end{verbatimtab}
Only the (left-)eigenvalue $1$ preserves the normalization property of
probability vectors and thus is the only one relevant. The corresponding eigenvector, appropriately normalized,
is the invariant probability vector.

A previous listing showed that
\marginpar{{\bf 3(a), 3(b)}}
$$
	P^1_{00} (00) = 0 \quad
  	P^2_{00} (00) = \frac{1}{4} \quad
	P^3_{00} (00) = \frac{1}{16} \quad
	P^4_{00} (00) = \frac{9}{64} = 0.140625
$$
The last value is, of course, the probability of returning to the original state $00$ after 4 steps.
Now, determine the probability $\PP_{revisit}$ that the chain revisits the state $00$ in the first 4 steps
\footnote{I have omitted the zero terms, such as $\PP(\textrm{revisit in 1})$ and $\PP(\textrm{revisit in 2 and again in 1})$}
\begin{equation}
\begin{split}
	\PP(\textrm{revisit in 4}) &= \PP (\textrm{revisit in 2}) 
			+ \PP(\textrm{revisit in 3}) 
			+ \PP(\textrm{revisit in 4}) \\
			& - \PP(\textrm{revisit in 2 steps and then in another 2 steps}) \\
		 &= P^2_{00}(00) + P^3_{00}(00) + P^4_{00}(00) - (P^2_{00}(00))^2 \\
		 &= \frac{1}{4} + \frac{1}{16} + \frac{1}{64} - (\frac{1}{4})^2 \\
		 &= \frac{25}{64} = 0.390625
\end{split}
\end{equation}

A numerical experiment we conduct in question 4 confirms these values, which is comforting.
\qed
\end{solution}

\appendix
\section*{Code Listing: Random Walk on a Graph}
% \input{random-walk-code}
\verbatimtabinput[4]{hw1q3.py}

\end{document}
